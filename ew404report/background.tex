\section{Background and motivation}\label{sec:background}
\IEEEPARstart{A}{nimals moving in a real environment} face many challenges that are currently unsurmountable for typical engineered devices, including small size, difficult-to-achieve power-to-weight ratios, ability to operate in multiple environments, and the ability to control complex maneuvers even in the presence of environmental disturbances such as wind, water flow, or turbulence, variable lighting, or even attack from predators or conspecific rivals. These alone make them worthy of engineering study, but such interdisciplinary work is not a one-way street. 

Biologists may also wish to use engineering to learn how an organism does what it does.  As an example, consider the aerodynamic principles of ``helicopter'' samara seeds such as in maples (\emph{Acer sp.}) and convergently evolved in many other groups.  Samaras are able to slow their descent to the group using a spanwise asymmetric weight distribution and wing structure to facilitate autorotation, allowing them longer hang time during which, on rare occasions, they are swept quite far by a lucky gust of wind. The resulting long dispersal distances are quite advantageous for the saplings, who no longer have to compete with the mother tree. The study of the first autorotating seeds in the fossil record made heavy use of engineering techniques such as autorotation theory, consideration of stability from the vertical separation of the center of gravity and the center of pressure, and nondimensional coefficients governing flight \cite{stevenson2015when}.  As for applications\footnote{Applications of basic biomechanics research may not be immediately obvious, but this is not a reason to ignore biomechanical systems of interest.}, researchers in \cite{who2019maple} developed a monocopter based off of this concept, and were able to develop control algorithms for actively steering such a device. 

Robotic systems are potentially a way to examine form and function in organisms, and organisms are potentially a source of inspiration for improving the robots.  In \cite{feltman2014creepy}, sidewinder-type locomotion, observed in all snakes but most characteristic of the sidewinder rattlesnake (\emph{Crotales cerastes}), was studied by creating a robot to imitate it. In sidewinder locomotion, snakes move across granular media like hot desert sand in a direction lateral to their main body axis; this is accomplished using a combination of body twisting and bending to create moving points of contact with the ground as the body bends.  The robot was developed like a snake, and was programmed to copy as closely as possible the movement of the sidewinder. The initial test had success on level ground, but no such luck on steeper slopes. On closer observation of the live snake, it was discovered that the sidewinder used two independent wavelike motions to move instead of just a single wave. Once this was implemented in the robot, the robot was able to navigate just like the sidewinder, and a new form of ground travel was made attainable by robotic machines. This discovery has applications of movement over a variety of different terrain that a typical wheeled robot would not be able to navigate.

I will be attempting to enhance a quadrotor’s maneuverability by studying the flight patterns and maneuvers of Anna’s Hummingbirds (\Calypteanna), a \SI{5}{\gram} hummingbird native to western North America. Male Anna's Hummingbirds perform a display dive in order to win matings with choosy females. The dive reaches speeds above \SI{27.3}{\meter\per\second} \cite{clark2009courtship}
% \cite{larimer1995accelerational}
 and ends with a 9G pullout maneuver (full 3D field kinematics determined in \cite{clark2009courtship}) culminating in a display of red iridescent feathers on his gorget (throat) and a high frequency tweeting noise made by flow-induced vibration of the distal two retrices (outer tail feathers) \cite{clark2008annas}. Such a maneuver would normally cause G-induced loss of consciousness in a human pilot without a g-suit.  The difficulty of the manuever makes it an honest signal \cite{zahavi1975mate} of mate quality to Anna's Hummingbird females as it requires skill and power to complete. Thus, it is expected to be difficult for a \SI{27}{\gram} quadrotor\footnote{The Crazyflie 2.1 mass is approximately the same as the largest living hummingbird species, the Giant Hummingbird (\emph{Patagona gigas}), native to the Andes.}  to imitate and a worthy challenge. I intend to discover how feasible the dive maneuvers actually are for quadrotors, and explore their operational limits in this respect. 

Unmanned aerial systems are becoming increasingly more autonomous for tasks with position control at comparatively low speeds; a guiding research question is how extreme maneuverability can augment their capabilities by building on previous work in quadrotor control  \cite{mellinger2011minimum, greiff2017modelling}.  The applications of this research are twofold: it will both help in developing a greater understanding of the physical limitations of extreme maneuverability on quadrotors, the primary engineering problem, and it will also provide a greater idea of the effects of such maneuvers on flying animals. The ability to automate the hummingbird dive maneuvers means that other forms of extreme maneuvers might also be autonomously executed, perhaps providing a toolbox or set of skills to use when a UAS is presented with a maneuvering challenge during a mission.  If combined with enhanced sensing abilities, autonomous MAVs would be able to navigate through obstacle-heavy environments with greater speed and ease, allow for evasion or penetration/infiltration, or simplify recovery. 

In addition to this, a successful device could be used in behavioral playback studies in which live hummingbirds are presented with a controllable stimulus mimicking the male display dive.  If a quadrotor were to execute this dive for a female Anna's Hummingbird, could the quadrotor generate a favorable response from her? If so, it may be possible to alter the pattern to present a super-normal stimulus or probe which aspects of the display are most appealing to her.  Such a study in hummingbirds would be novel\footnote{von Frisch, Lorenz, and Tinbergen won the 1973 Nobel Prize for Physiology or Medicine for ``discoveries concerning organization and elicitation of individual and social behaviour patterns...'' among other things, including use of robots to examine honeybee dance language and supernormal stimuli in herring hull beak markings.}  as biologists have not been able to produce the maneuvers themselves.
