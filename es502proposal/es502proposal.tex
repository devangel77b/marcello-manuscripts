\documentclass{IEEEtran}

\usepackage{graphicx}
\usepackage{svg}
\usepackage{siunitx}
\newcommand{\myroot}{../}
\newcommand{\Gensp}[1]{\emph{#1}}
\newcommand{\Hirudomedicinalis}{\Gensp{Hirudo~medicinalis}}

\title{Bio-inspired soft robot}
\author{M.~Descour, L.~Devries, and D.~Evangelista\thanks{Authors are with the United States Naval Academy, Department of Weapons \& Systems Engineering}}
\date{\today}

\begin{document}
\maketitle

\begin{abstract}
Soft robotics provide a solution to the lack of maneuverability, durability, and degrees of freedom illustrated by many traditional hard-bodied robots. Challenges of soft robotics include their actuation and controllability. By examining the locomotion of a leech over land, this research proposes a novel method of soft robotic control. Specifically, this research will focus on soft pneumatic actuators, and their subsequent manipulations to accomplish locomotion. Research will be conducted to determine a system that optimally illustrates predetermined properties. The properties are the ability of forward motion, the incorporation of couple attachment points for variable environment maneuverability, and open loop control. The demonstration plan of this research will consist of three separate proof-of-concept demonstrations. A proof-of-concept demonstration will focus on the bending pneumatic actuator, determining the relationship between input pressure and bend angle as well as speed of actuation. The second proof-of-concept demonstration will determine a viable method for attachment/detachment that may also be incorporated into the soft actuator. The final proof-of-concept demonstration will be contingent upon the success of the previous two demonstration experiments. The final demonstration will examine the feasibility of locomotion, combining the previously acquired data. The total cost of this research is \$28,235 including materials, labor, and overhead. The research plan includes risk mitigation related to possible technical failures. This risk mitigation includes the design and fabrication portions of the research occurring in the early fall semester to allow adequate time for data collection.
\end{abstract}
     
\section{Background and Motivation}
\IEEEPARstart{T}raditionally, robots are thought of as having rigid, metallic bodies with discrete joints and hard material composition.  However, these rigid bodies may have difficulties in manipulation and maneuverability, generally offering limited degrees of freedom On the other hand, many natural organisms have bodies that are soft and flexible, with the ability to deform in various ways.  Engineers have turned to biology as a source of inspiration for robotic designs.  Soft, bio-inspired robots offer limitless degrees of freedom, allowing opportunities to bend, twist, expand, and contract in various ways \cite{rus2015design}.  The multi-gait robot shown in Figure~\ref{f1} illustrates some of the potential advantages of soft robots such as maneuverability combined with resiliency \cite{shepherd2011multigait}.
\begin{figure}[hb]
\begin{center}
\includegraphics[width=0.5\columnwidth]{\myroot/figures/proposal1.jpg}
\end{center}
\caption{The ``Resilient, Untethered Soft Robot'' \cite{shepherd2011multigait}}
\label{f1}
\end{figure}

The challenges of soft robotics include their actuation and subsequent controllability.  As soft robotic designs lack many of the hard components such as servos and motors found in traditional machines, unique methods of actuation are incorporated.  For soft robots, researchers have focused on two primary methods.  In one method, variable length tendons, such as shape memory alloys, are embedded in soft materials.  The other method, that will be discussed and researched in this paper, is pneumatic actuation.  Pneumatic actuation for soft robotic systems was first explored in 1992, in which channels in an elastomer were inflated via pressurized air.  In this design, constructed asymmetry caused the actuator to move in a desired manner, shown in Figure~\ref{f2} \cite{tanaka1992applying}.  More recently, researchers have manipulated actuation of a soft robot with the implementation of an inextensible layer, allowing variable bending based on input pressure as shown in Figure \cite{polygerinos2015modeling}.
\begin{figure}
\begin{center}
\includegraphics[width=0.5\columnwidth]{\myroot/figures/proposal2.png}
\end{center}
\caption{A 1992 soft robotic microactuator achieving desired movement due to asymmetry in the design \cite{tanaka1992applying}}
\label{f2}
\end{figure}

With the current research and developments into soft robotics, there exists a number of different real-world societal applications. The maneuverability of soft robotics allows for their potential implementation in otherwise difficult environments.  Researchers have looked into robots capable of navigating obstacles such as rubble in search and rescue operations of natural disasters, such as earthquakes and hurricanes\cite{irv2017xxx}.  Furthermore, due to the soft nature of the materials, soft robotic systems are used in the medical field as wearable applications.  Soft robotics generally offer more comfort for human-use applications, as rigid metallic devices have the risk of causing damage to human tissue.  Examples of medical applications include wearable devices for orthopedic rehabilitation \cite{par2014xxx} and soft sensing suits for lower limb measurement \cite{men2014xxx}.

Additionally, the capability to operate robotics in an underwater environment could be improved through the implementation of soft actuators due to increased maneuverability in difficult spaces.  Focusing on the underwater environment, soft robotic research can be expanded to fit real-world military applications.  Notably, the United States military notably uses autonomous underwater vehicles (AUVs) to accomplish missions in Intelligence, Surveillance, and Reconnaissance (ISR), and has pledged \$600 million in their development \cite{pomerleau2016dod}.  Applications of a bio-inspired and highly maneuverable soft robot are not limited to conducting ISR at sea, and may be expanded to usage on land and littoral environments.

\section{Problem statement}
The medicinal leech (\Hirudomedicinalis) is capable of crawling movements, achieved via the sequential activation of muscle groups arranged in longitudinal segments and coupled attachment points at the anterior and posterior ends, as shown in Figure~\ref{f3} \cite{kristan2005neuronal}.  

Pneumatic actuators, as shown in Figure~\ref{f4}, are capable of variable bending under pressure inputs \cite{polygerinos2015modeling}.  The device in Figure~\ref{f4} consists of chambers embedded within Ecoflex silicon material that bend about an inextensible bottom layer upon pressurization.  

While \cite{polygerinos2015modeling} examined the effects of pressure inputs on the bending of soft actuators, a method for locomotion was not considered.  I propose to take inspiration from leech-like crawling to examine if a soft bending pneumatic actuator can accomplish locomotion via attachable ends.  This research proposes a bio-inspired, pneumatically actuated, soft robot and control laws capable of the following properties:
\begin{enumerate}
\item The soft robot is able to move forward in a ``crawling'' motion, as well as turn via a twisting motion.
\item The soft robot incorporates bio-inspired coupled attachment points, allowing maneuverability in variable environments.
\item Control will be open loop.  The input of the control law will be desired motion and the output will be pressure and subsequent shape modifications necessary for the desired motion.
\end{enumerate}

\begin{figure}
\caption{The successive stages of leech crawling \cite{kristan2005neuronal}}
\label{f3}
\begin{center}
\includegraphics[width=0.5\columnwidth]{\myroot/figures/proposal3.png}
\end{center}
\end{figure}

\begin{figure}
\caption{A soft pneumatic acutator bending under variable pressure inputs \cite{polygerinos2015modeling}}
\label{f4}
\begin{center}
\includegraphics[width=0.5\columnwidth]{\myroot/figures/proposal4.png}
\end{center}
\end{figure}


\section{Literature review}
In the scientific community, the medicinal leech (\Hirudomedicinalis) has served as an extensively studied organism in the fields of neuroscience and biology.  This can be attributed to the relative simplicity of leech anatomy and leech behavior, both of which are discussed in \cite{kristan2005neuronal}.  Researchers in \cite{kristan2005neuronal} have produced detailed descriptions of six common leech mechanisms: heartbeat, local bending, shortening, swimming, crawling, and feeding.  Most relevant to the research of a bio-inspired soft robot are the locomotive behaviors of bending, shortening, swimming, and crawling.  In \cite{kristan2005neuronal}, researchers documented the relationship between the circular and longitudinal muscles that enable these locomotive functions.  With respect to crawling, leeches exhibit both vermiform crawling (extension and contraction of hydrostatic skeleton) and ``inch-worm'' crawling (similar to vermiform crawling, but the suckers are brought adjacent to each other at the end of each contraction).  The researchers noted the greater efficiency of ``inch-worm'' crawling, despite its rarer implementation due to the natural instability of the leech.

Due to the simple hydrostatic skeletal structure of a leech, researchers in \cite{alscher1998simulating} have developed a mathematical model for the dynamical behavior of a leech.  The model closely follows leech anatomy, including twenty-one compartmentalized segments, each with a fixed volume.  The circular and longitudinal muscle movements are modeled by elastic edges acting as damped elastic springs.  Through these constraints, researchers developed equations of motion modeling the pressure values in the compartmentalized segments and, subsequently, the leech movements themselves.  The dynamic leech model was successfully simulated, demonstrating its capability to generate the leech movements of crawling and swimming \cite{alscher1998simulating}.  The model is limited by a lack of experimental data to determine its validity, yet it provides a method in which the major constructional principles of the leech may be mathematically analyzed. 

Soft robotics provide a solution to developing an experimental platform in which leech locomotion may be replicated.  Soft robots, heavily inspired by nature, are composed of compliant materials with deformable bodies.  The most common methods of soft robotic actuation are variable length tendons embedded in soft segment and pneumatic actuation to inflate embedded channels in soft material. In \cite{rus2015design}, the many challenges of soft robotics are identified, including controlling soft materials that bend, twist, and stretch, offering infinite degrees of freedom.  Another identified challenge of soft robots is the implementation of power sources for actuation.  Currently, power sources for pneumatically actuated soft robots are limited to pumps or compressed air cylinders, both of which are bulky and may potentially inhibit the maneuverability of the robot. 

Despite these challenges, this research is most interested in pneumatic actuation for soft robots, due to its affordability and its customizability to a given application.  Researchers at Harvard University have modeled, designed, and tested soft pneumatic actuators, analyzing the effect of input pressure to various outputs \cite{polygerinos2015modeling}.  Two different models were designed for the analysis of the fabricated soft actuator.  An analytical model was developed to define the relationship between the input pressure and bending angle in free space, using material and geometric properties of the actuator.  In order to bypass some of the limits of the analytical model, a finite-element method model was also developed to model the nonlinear responses of the actuator at unpressurized and pressurized states.  An experimental platform was constructed to validate the analytical and FEM models.  In addition, the controllability of the actuator was illustrated through a feedback control loop embedded in the actuator that calculated bending angle from air pressure in real time.

Inflating pneumatic networks (``pneu-net''), or embedded small channels in soft elastomeric materials, allow for sophisticated motions with simple controls and inputs.  At Harvard University, researchers focused on improving existing pneumatic networks seen in \cite{polygerinos2015modeling} for speed and overall efficiency \cite{mosadegh2014pneumatic}.  Using silicone-based elastomers, the newly designed actuators were empirically tested and demonstrated.  The new pneumatic network design features multiple advancement including a higher speed for inflation, greater force exerted under a given pressure, lower change in volume for given degree of bending, and higher resiliency before failing.  Their research is applicable as it optimizes a pneumatic network under the constraint of limited resources, such as lower operating pressures, smaller volumes, and smaller time constraints.  In addition to comparisons with old pneumatic actuators, he newly designed pneumatic network demonstrated its speed and precision by playing notes on an electronic keyboard in succession, emulating human fingers \cite{mosadegh2014pneumatic}. 

In response to the challenges presented in \cite{rus2015design} of the implementation of bulky power sources that restrict maneuverability, another team of researchers at Harvard University designed a completely untethered variant of a previously tethered multi-gait soft robot \cite{shepherd2011multigait} \cite{tolley2014resilient} .  The untethered soft robot moves freely and is able to carry its own weight, including power source, over a substantial period.  The soft robot also maintains its resiliency, previously seen in its tethered variant.  Resiliency was experimentally tested under the exposure of a flame, run over a car, and walking outside in a snowstorm.  The untethered and tethered variations of this soft robot demonstrate an ability to achieve ``walking'' locomotion through pneumatic actuation.  Locomotion is accomplished via a four ``legged'' structure, with five total actuators acting in combination.  The researchers noted the lack of optimization in the design of this particular soft robot, as the overall actuation speed and mobility could see improvement. 

As shown in \cite{shepherd2011multigait} and \cite{tolley2014resilient}  soft robots are capable of locomotion through pneumatic actuation.  This research seeks to improve these methods of soft robot locomotion through inspiration taken from leech behavior discussed in \cite{kristan2005neuronal}.  Specifically, leech ``inch-worm'' crawling, through the unique use of ``suckers'', has demonstrated its ability to be mathematically modeled in \cite{alscher1998simulating}.  By implementing the leech’s ability to adhere to surfaces at both ends of its body structure to a soft robot, locomotion may be achieved.  This research focuses on cutting down the number of actuators as seen in \cite{shepherd2011multigait} and \cite{tolley2014resilient}, while still accomplishing locomotion.  The research done in \cite{polygerinos2015modeling} serves to provide insight on the effect of input pressure to the bending angle of a pneumatic actuator, as bending is a key component of the leech ``inch-worm'' crawl.




\section{Demonstration plan}
The demonstration plan for this research will consist of three separate proof-of-concept demonstrations.  The proof-of-concept demonstrations allow for the design and creation of the actual pneumatic actuator, as well as determining a method to best accomplish attachment and detachment to reflect leech behavior.  The demonstrations of the pneumatic actuator itself and the attachment/detachment mechanisms will be independent of one another, thus allowing for continuation of research given obstacles in the demonstration process.  Ultimately, the goal is for a final proof-of-concept experiment of accomplishing locomotion of the actuator.  This will require success in both the of the previous proof-of-concept demonstrations as both bending actuation and coupled attachment/detachment are necessary for the desired locomotion.  The proof-of-concept demonstrations will display the properties of simplicity, replicability, and maneuverability.

\subsection{Proof-of-concept experiment: bending actuation}
The proof-of-concept demonstration for a bending actuator requires the construction of an electrical circuit as well as the design of the soft actuator itself.  The circuit will be designed using guidance from the Soft Robotics Toolkit \cite{holland2014soft}, where the specific electrical components may also be found.  In the circuit, a desired pressure output is sent to microcontroller as a voltage value, which is then sent as data to the air pump.  Air travels through the solenoid valve and pressurizes the soft actuator.  The soft actuator will inflate, and a bending angle will be observed. A pressure reading from a pressure sensor will also be measured.  Position and bend angle of the actuator will be recorded and visually interpreted.  Graph paper will be placed behind the soft actuator to aid in determining the bend angle, determining how the pressure inputs relate to bend angle for the specific actuator.
	
The second aspect of the proof-of-concept experiment for the bending actuator will be the design and fabrication of the soft actuator itself.  The intent is to create two different kinds of soft actuator, the PneuNet actuator (shown in Figure~\ref{f7}) and the soft fiber-reinforced actuator (shown in Figure~\ref{f8}).  Existing designs found in the Soft Robotics Toolkit \cite{holland2014soft} will be used in the fabrication process.  Both designs required 3-D printed molds, filled with silicon material.  The purpose of designing two different kinds of soft actuator will be for comparison of respective performance of simplicity, durability, and manipulability to fit the purposes of this research (i.e. incorporation of an attachment/detachment mechanism).  Specifically, a measured metric will be the time required to reach a specific bend angle. Smaller times are desired, as this signifies increased speed and maneuverability of the final system.   

\begin{figure}
\begin{center}
\includegraphics[width=\columnwidth]{\myroot/figures/proposal5.png}
\end{center}
\caption{Circuit, from \cite{holland2014soft}, that will be implemented to provide pressure to the actuator}
\label{f5}
\end{figure}

\begin{figure}
\begin{center}
\includegraphics[width=\columnwidth]{\myroot/figures/proposal6.png}
\end{center}
\caption{Functional block diagram of the soft actuator}
\label{f6}
\end{figure}

\begin{figure}
\begin{center}
\includegraphics[width=\columnwidth]{\myroot/figures/proposal7.png}
\end{center}
\caption{Building schematic of a PneuNet actuator, from \cite{holland2014soft}}
\label{f7}
\end{figure}

\begin{figure}
\begin{center}
\includegraphics[width=0.75\columnwidth]{\myroot/figures/fabricationprocess.png}
\end{center}
\caption{Building schematic of a fiber-reinforced actuator \cite{holland2014soft}}
\label{f8}
\end{figure}

\subsection{Proof-of-concept experiment: attachment and detachment methodology}
The next proof-of-concept demonstration will be the design of an attachment and detachment mechanism that can be incorporated into the soft actuator.  Research, as well as several designs, will need to be conducted for this proof-of-concept demonstration.  The primary design that will be investigated is illustrated in the sketch shown in Figure~\ref{f9}.  Vacuum chambers will be incorporated into the soft actuator allowing for suction, and subsequently a method for attachment.  This design will require incorporating vacuum pumps into the circuitry described earlier in Figure~\ref{f3}.  Other methods of an attachment mechanism will be explored through prototype and fabrication such as ``grippers'' used in the RiSE robot illustrated in Figure~\ref{f10}, as well as other bio-inspired adhesives such as starfish feet and non-Newtonian fluids excreted by gastropods.  The designs will be tested on surfaces of variable composition and incline. The performance of the designs will be visually interpreted as well as video recorded.  The performance of the designs will be compared based on the properties of repeatability, simplicity, and ability to be incorporated into the soft actuator. The designs will also be compared based on the strength of the suction force. To measure this metric, a push pull force gauge will be attached to the suction mechanism and the force required for detachment will be calculated.  Higher forces required for detachment are preferable as this will correleate to the maneuverability of the final system, specifically in varied environments. 

\begin{figure}
\begin{center}
\includegraphics[width=0.5\columnwidth]{\myroot/figures/proposal9.png}
\end{center}
\caption{Sketch of the incorporation of a vacuum induced attachment and detachment into the soft actuator.}
\label{f9}
\end{figure}

\begin{figure}
\begin{center}
\includegraphics[width=0.5\columnwidth]{\myroot/figures/proposal10.png}
\end{center}
\caption{RiSE Gecko robot capable of adhering to multiple surfaces \cite{RiSEphoto}}
\label{f10}
\end{figure}

\subsection{Proof-of-concept experiment: locomotion}
The final proof-of-concept demonstration requires successful demonstrations of the soft bending actuator as well as a working attachment and detachment method.  Locomotion will require comprehension of the interaction between the bending of the soft actuator and the attachment method.  For example, a specific bend-angle of the actuator may be necessary to initiate an attachment to a surface inclined at a specific angle.  Figure~\ref{f11} illustrates a proposed method to achieve leech-like locomotion with the soft actuator.  Locomotion will require the ability to increase and decrease input pressure to change the bend angle ($\theta$) to a desired state.  As input pressure increases, the bend angle will increase.  In the proposed method, the robot will alternate between a small bend angle ($\theta_S$) and a large bend angle ($theta_L$) to achieve locomotion.  During this process, the forward and rear ends of the robot will also alternate between an attached and detached state.  This proposed method of locomotion is solely for the purpose of producing a forward movement of the actuator in a straight line.
\begin{figure}
\begin{center}
\includegraphics[width=\columnwidth]{\myroot/figures/proposal11.png}
\end{center}
\caption{A proposed method for soft robot locomotion over seven stages}
\label{f11}
\end{figure}

\subsection{Time risks and mitigation}
The major time risks associated with this project involve the design and fabrication of the soft actuators as well as the attachment/detachment mechanism.  Building and modifying an electro-pneumatic circuit for the purposes of this project also poses a time risk.  In order to mitigate this time risk, a basic electro-pneumatic circuit will be completed as soon as possible.  This basic circuit will have the ability to inflate and deflate a simple actuator, while providing the ability to be modified to incorporate additional air pumps or vacuum pumps as needed.

\subsection{Technical risks and mitigation}
Due to the novelty of the proposed attachment/detachment mechanism, there exists a technical risk of an ideal vacuum ``gripper'' not working at all.  Another technical risk is the possibility of damaging the soft actuators during experimentation or improper handling.  To mitigate this risk, multiple copies of the actuators will be built.  Furthermore, construction of the soft actuators will occur immediately at the start of the fall semester.

\subsection{Justification of special high risk activities}
This project involves the purchase and care-taking of live leeches and earthworms (\emph{Lumbricus sp.}) for feeding them.  There exists a lower risk as these organisms are invertebrates and can be cared for with relative ease.  The leeches and earthworms will aid in the demonstration of this proposal, as the performance and components of the soft actuator will be compared with that of actual annelids that are the source of the bio-inspiration. 

\subsection{Budget}
\begin{table}
\caption{Budget}
\label{tbudget}
\begin{center}
\includegraphics[width=\columnwidth]{\myroot/tables/proposalbudget.png}
\end{center}
\end{table}

\section{Conclusion}
By examining leech locomotion on land, a novel method of bio-inspired soft robotic locomotion can be developed. This research focuses on accomplishing this system of locomotion via pneumatic actuators. Through three separate proof-of-concept demonstrations, the properties of the bio-inspired pneumatic actuator will be illustrated and analyzed. These properties include forward movement, inclusion of coupled attachment points, and open loop control. Successful research of the proposed system will aid in developing novel methods of search and rescue (SAR) and intelligence, surveillance, and reconnaissance (ISR) for real-world applications. As proof-of-concept demonstrations will be conducted in this research, the biggest risk are technical failures of the proposed designs. To mitigate this risk, the design and fabrication portions of the research will be conducted early in the fall semester to ensure the possibility of data collection.

\appendix
%Gantt chart here.
\begin{table*}
\caption{Gantt chart}
\label{tgantt}
\begin{center}
\includegraphics[width=\columnwidth]{\myroot/tables/proposalgantt.png}
\end{center}
\end{table*}



\bibliographystyle{IEEEtran}
\bibliography{IEEEabrv,\myroot/references/descour}
\end{document}
