\section{Literature review}
\label{sec:literature}

%The inspiration of this project proposal has its roots in the high maneuverability seen by hummingbirds. As such, it was essential to be able to gather data on hummingbird trajectories in order to study their feasibility of being replicated by a quadrotor. There have been many bio researchers that have delved into the study of hummingbirds, however I will mainly discuss the works \cite{clark2009courtship} and \cite{cheng2016flight}. 

The work in \cite{cheng2016flight} determined the trajectory and body kinematics of four different hummingbird species in an evasive maneuver. This was done by startling the birds while they were hovering, and observing their movements with three high definition cameras to provide a 3D position \cite{hedrick2008software, theriault2014protocol, jackson20163d}. Birds were marked with water-soluble white paint to ease digitization and tracking. I will use a similar optical tracking methods as in \cite{cheng2016flight, hedrick2008software, theriault2014protocol, jackson20163d}, with the added simplifications of being able to install known, infrared (IR) relfective markers for automatic tracking and using a RANSAC algorithm to estimate pose. In addition, the evasive maneuvers studied in \cite{cheng2016flight, sholtis2015field} could be useful for quadrotors attempting to avoid capture or a counter-UAS drone denial system (e.g. USNA Project Midknight or similar).

% Water-soluble white paint was used to make dot markings on the hummingbird’s body to help model the wing and head positions for each trial. The data they acquired in their experiments includes many more details than I will need to use, however they provide data on the hummingbird’s velocity and trajectories for several trials which I can use to help develop my quadrotor trajectories. , they and I will both be using optical data to obtain a position fix. With regard to the type of maneuver that these birds are required to perform, it aligns well with the type of experimentation that I aim to work with. Evasive maneuvers are certainly a type of extreme maneuver, and these patterns may prove to be something that I wish to try on my quadrotor. Evasive maneuvers can be useful for quadrotors if they are in threat of being netted, and if I am able to copy the hummingbird’s trajectory, further analysis and testing may prove that this type of trajectory provides a maneuvering or sensing advantage to the quadrotor in evasive flight.

The work in \cite{clark2009courtship} obtained fully 3D field kinematics of courtship dives in \Canna\ in order to study extreme locomotor performance in animals. \cite{clark2009courtship} used multiple calibrated high speed cameras and manual digitization to reconstruct the birds' 3D position during dives.  Splines were used to estimate acceleration and velocity from positions without undue amplification of noise. \cite{clark2009courtship} also examined wing and tail movements and sounds produced during dives.  As a measure of the biomechanical difficulty of the maneuver, \cite{clark2009courtship} estimated the maximum stresses in the humerus during dive pullout; such a quantity would be extraordinarily difficult to measure \emph{in vivo} but in  my work there is a possibility of directly instrumenting the quadrotor to examine forces, torques, stresses, and engineering limits. 
%Again, this is more detailed data than I will need for trajectory replication on my quadrotor. This paper gives an insight into what exactly I will be trying to achieve through the extreme maneuverability of my quadrotor. Since this type of maneuver is estimated to cause a lot of strain on the hummingbird, it is likely to also cause a lot of strain on a quadrotor. Through simulation and proof of concept demonstration, I will be able to provide a more accurate picture of just how difficult these maneuvers can actually be.

After obtaining these hummingbird trajectories, an effective method of modeling a quadrotor and conducting trajectory planning is required. In \cite{tomic2014learning}, the authors developed metrics and constraints for quadrotor maneuvering performance and a machine learning workflow. This was achieved through the solving of an optimal control problem offline, and then using a machine learning technique to learn these trajectory solutions with the given constraints. The result was then used to develop online solutions  for near-optimal trajectories for a quadrotor. This was done in the $x$-$z$ plane for point to point and perching maneuvers, as well as joint trajectories. To validate their solution, they flew these optimal trajectories using both Simulink simulations, and proof of concept demonstrations. Since I will be using a quadrotor platform, this paper directly applies to my problem statement as a good reference base that I can use to springboard my exploration into more complex extreme maneuvering. The basis of this work will give me a much more quantitative measure of success in terms of how close my developed trajectories are to an optimal path. \cite{tomic2014learning} is very thorough and provides a clear distinction and improvement on previous work in quadcopter trajectories, especially with regard to the joint trajectory problem. I can build on this by expanding into 3D trajectories instead of just working in a 2D plane, and I can also try to utilize their proxy-based joining method to create a desired path curvature.

In \cite{sabatino2015quadrotor}, the authors developed a linearized model of a quadrotor in planar motion. The careful process by which the quadrotor dynamics are identified and modeled will be helpful in my own research as I develop my own model for the quadrotor that I will be using. In \cite{sabatino2015quadrotor}, their modeling method is done for three different linearization methods and each of these is compared to each other by running a Simulink simulation with each controller. Quantities compared include several attributes of the step response, and the actual trajectory of the quadrotor compared to the desired trajectory. This comparison method between the different trajectories is similar to the validation work that I will need to do on my own simulation. As such, this work will help me to better understand ways of determining the accuracy of my trajectory testing in simulation, and in proof of concept demonstration. This paper, while a good starting point for my work, does not attempt to go into more complex maneuvers. These are discussed in greater detail in the following works.

In \cite{liu2017planning}, the development of trajectories and path planning for UAVs was accomplished. They did this by determining the maximum overload, minimum turn radius, and maximum flight endurance of the experimental quadrotors in order to come up with feasible aggressive trajectories. Trajectories had the constraint that they had to follow a sixth order (or lower) polynomial trajectory. Much like my proposed concept, this work develops an attitude and trajectory controller with appropriate initial and final conditions, as well as a boundary “tube” which the quadrotor must stay within for every trajectory. The work in this project is heavily relevant to my proposed work, as they achieve a working simulation of aggressive trajectories with their path-planning algorithm and onboard controllers. I would like to expand on this work by flying a shorter trial with hummingbird-like flight patterns. 

Finally, \cite{mellinger2011minimum} offers some of the closest work to exactly what I am proposing for my own work. The main focus of this paper is to create trajectories for quadrotors in real time in an indoor or constrained environment. They also pay particular attention to the velocity and acceleration vectors of the quadrotor throughout its maneuver. I will also need to be able to achieve these types of measurements from my system, and be able to change my controller to affect them in an appropriate manner in order to fully achieve a trajectory flight path that replicates a hummingbird maneuver. \cite{mellinger2011minimum} also uses temporal scaling to fly their trajectories at different speeds, which is exactly what I will need to do when and if I find that flying the hummingbird trajectory at full speed is either not possible or extremely dangerous. 
